From the given information, 
\begin{align}
\vec{P} &= \myvec{3&-2}\myvec{\vec{a}\\\vec{b}}\\
\vec{Q} &= \myvec{1&1}\myvec{\vec{a}\\\vec{b}}
\\
\implies \myvec{\vec{P}\\\vec{Q}} &= \myvec{3 & -2\\1 & 1}\myvec{\vec{a} \\ \vec{b}}
\end{align}
\begin{enumerate}
\item 
For  internal division, using section formula, 
\begin{align}
\vec{R} &= \myvec{\frac{m}{m+n} & \frac{n}{m+n}}\myvec{\vec{P}\\\vec{Q}}\\
&= \myvec{\frac{m}{m+n} & \frac{n}{m+n}}\myvec{3 & -2\\1 & 1}\myvec{\vec{a} \\ \vec{b}}
\end{align}
For ratio 2 : 1,
\begin{align}
\vec{R} &= \myvec{\frac{2}{2+1} & \frac{1}{2+1}}\myvec{\vec{P}\\\vec{Q}}\\
&= \myvec{\frac{2}{3}&\frac{1}{3}}\myvec{3 & -2\\1 & 1}\myvec{\vec{a} \\ \vec{b}}\\
&= \myvec{\frac{7}{3} & -1}\myvec{\vec{a} \\ \vec{b}}\\
\implies \vec{R} &= \frac{7}{3}\vec{a} - \vec{b}
\end{align}

\item Similarly,  for external division,
\begin{align}
\vec{R} &= \myvec{\frac{m}{m-n} & \frac{n}{m-n}}\myvec{\vec{P}\\\vec{Q}}\\
&= \myvec{\frac{m}{m-n} & \frac{n}{m-n}}\myvec{3 & -2\\1 & 1}\myvec{\vec{a} \\ \vec{b}}
\end{align}
For ratio 2 : 1,
\begin{align}
\vec{R} &= \myvec{\frac{2}{2-1} & -\frac{1}{2-1}}\myvec{\vec{P}\\\vec{Q}}\\
&= \myvec{2 & -1}\myvec{3 & -2\\1 & 1}\myvec{\vec{a} \\ \vec{b}}\\
&= \myvec{5 & -5}\myvec{\vec{a} \\ \vec{b}}\\
\vec{R} &= 5\vec{a} - 5\vec{b}
\end{align}
%
\end{enumerate}