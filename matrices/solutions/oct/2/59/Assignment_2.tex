
The Cayley-Hamilton theorem states that any $N\times N$ matrix satisfies it's characteristic equation.  The characteristic equation of a matrix A is 
\begin{align}
    |{A - \lambda I}| = 0\label{oct/2/59/eq 2.0.1}\\
    & \implies \mydet{\myvec{3 & -2 \\ 4 & -2} - \lambda\myvec{1 & 0 \\ 0 & 1}} = 0\\
    & \implies \mydet{3 - \lambda & -2 - \lambda \\ 4 & -2-\lambda} = 0\\
    & \implies \brak{3-\lambda}\brak{-2-\lambda} - 4\brak{-2-\lambda} = 0\\
    & \implies \lambda^2 -\lambda +2 = 0 \label{oct/2/59/eq 2.0.5}\\
\end{align}

From \eqref{oct/2/59/eq 2.0.5} and \eqref{oct/2/59/eq 2.0.1}
\begin{align}
    A^2 - A +2I = 0
\end{align}

As we compare the given equation with the equation acquired on solving, we get
\begin{align}
    k = 1
\end{align}
