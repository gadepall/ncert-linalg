\documentclass{beamer}
\usepackage{listings}
\lstset{
%language=C,
frame=single, 
breaklines=true,
columns=fullflexible
}
\usepackage{blkarray}
\usepackage{subcaption}
\usepackage{url}
\usepackage{tikz}
\usepackage{tkz-euclide} % loads  TikZ and tkz-base
%\usetkzobj{all}
\usetikzlibrary{calc,math}
\usepackage{float}
\newcommand{\myvec}[1]{\ensuremath{\begin{pmatrix}#1\end{pmatrix}}}
\newcommand{\mydet}[1]{\ensuremath{\begin{vmatrix}#1\end{vmatrix}}}
\providecommand{\brak}[1]{\ensuremath{\left(#1\right)}}
\newcommand\norm[1]{\left\lVert#1\right\rVert}
\renewcommand{\vec}[1]{\mathbf{#1}}
\usepackage[export]{adjustbox}
\usepackage[utf8]{inputenc}
\usepackage{amsmath}
\usepackage{physics}
\usepackage{tikz}
\usetikzlibrary{automata, positioning}
\usetheme{Boadilla}
\providecommand{\pr}[1]{\ensuremath{\Pr\left(#1\right)}}

\title{Assignment 2 Presentation}
\author{Akyam L Dhatri Nanda}
\date{AI20BTECH11002}
\begin{document}
\begin{frame}
\titlepage
\end{frame}

\begin{frame}
\frametitle{Question}
\begin{block}{Matrix Q.2.59}
If A = $\myvec{3 & -2 \\ 4 & -2}$ and I = $\myvec{1 & 0 \\ 0 & 1}$, find k so that $A^2 = kA -2I$
\end{block}
\end{frame}

\begin{frame}
\frametitle{}
\begin{block}{Theorem}
The Cayley-Hamilton theorem states that any $N\times N$ matrix satisfies it's characteristic equation.
\end{block}

\begin{block}{Characteristic equation of a matrix}
For any square matrix M, the characteristic equation is $|{M - \lambda I}| = 0$
\end{block}
\end{frame}
\begin{frame}
\frametitle{Solution}
The characteristic equation of Matrix A is 
\begin{align}
    |{A - \lambda I}| = 0\label{eq 2.0.1}\\
    & \implies \mydet{\myvec{3 & -2 \\ 4 & -2} - \lambda\myvec{1 & 0 \\ 0 & 1}} = 0\\
    & \implies \mydet{3 - \lambda & -2 - \lambda \\ 4 & -2-\lambda} = 0\\
    & \implies \brak{3-\lambda}\brak{-2-\lambda} - 4\brak{-2-\lambda} = 0\\
    & \implies \lambda^2 -\lambda +2 = 0 \label{eq 2.0.5}
\end{align}

From Cayley-Hamilton theorem, 
\begin{align}
    A^2 - A +2I = 0\\
    \implies k=1
\end{align}

\end{frame}
\end{document}
