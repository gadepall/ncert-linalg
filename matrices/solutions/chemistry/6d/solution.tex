Let the balanced version of \eqref{eq:solutions/chem/6d1} be:-
\begin{align}\label{eq:solutions/chem/6d3}
x_1BaCl_2 + x_2H_2SO_4 \xrightarrow{} x_3BaSO_4 + x_4HCl
\end{align}
which results in the following equations:-
\begin{equation}
 \begin{aligned}
    (x_1-x_3)Ba=0\\
    (x_2-x_4)SO_4=0\\
    (2x_2-x4)H=0\\
    (2x_1-x_4)Cl=0
 \end{aligned}
\end{equation}
which can be expressed as:-
\begin{equation}
 \begin{aligned}
    1.x_1 + 0.x_2 - 1.x_3 + 0.x_4=0\\
    0.x_1 + 1.x_2 + 0.x_3 - 1.x_4=0\\
    0.x_1 + 2.x_2 + 0.x_3 - 1.x4=0\\
    2.x_1 + 0.x_2 + 0.x_3 - 1.x_4=0
 \end{aligned}
\end{equation}
resulting in the matrix equation:-
\begin{align}\label{eq:solutions/chem/6d2}
    \myvec{1 & 0 & -1 & 0\\0 & 1 & 0 & -1\\0 & 2 & 0 & -1\\2 & 0 & 0 & -1}\vec{x}=\vec{0}
\end{align}
where
\begin{align}
    \vec{x}=\myvec{x_1\\x_2\\x_3\\x_4}
\end{align}
equation \eqref{eq:solutions/chem/6d2} can be row reduced as follows
\begin{align}
   \myvec{
 1 & 0 & -1 & 0\\  
 0 & 1 & -1 & 0\\
 0 & 2 & 0 & -1\\
 2 & 0 & 0 & -1
}\xleftrightarrow[]{R4\leftarrow R4-2R1}\myvec{
1 & 0 & -1 & 0\\
0 & 1 & -1 & 0\\
0 & 2 & 0 & -1\\
0 & 0 & 2 & -1}
\end{align}
\begin{align}
   \myvec{
1 & 0 & -1 & 0\\
0 & 1 & -1 & 0\\
0 & 2 & 0 & -1\\
0 & 0 & 2 & -1}\xleftrightarrow[]{R3\leftarrow R3-2R2} \myvec{
1 & 0 & -1 & 0\\
0 & 1 & -1 & 0\\
0 & 0 & 2 & -1\\
0 & 0 & 2 & -1}
\end{align}
\begin{align}
    \myvec{
1 & 0 & -1 & 0\\
0 & 1 & -1 & 0\\
0 & 0 & 2 & -1\\
0 & 0 & 2 & -1}\xleftrightarrow[]{R4\leftarrow R4-R3} \myvec{
1 & 0 & -1 & 0\\
0 & 1 & -1 & 0\\
0 & 0 & 2 & -1\\
0 & 0 & 0 & 0}
\end{align}
Thus,
\begin{align}
    x_1=x_3,x_2=x_3,2x_3=x_4\\
    \vec{x}=x_3\myvec{1\\1\\1\\2}\label{eq:solutions/chem/6d4}
\end{align}
Upon substituting $x_3=1$ in\eqref{eq:solutions/chem/6d4},then \eqref{eq:solutions/chem/6d3} becomes,
\begin{align}\label{eq:solutions/chem/6d5}
    \boxed{BaCl_2 + H_2SO_4 \xrightarrow{} BaSO_4 + 2HCl}
\end{align}
 \eqref{eq:solutions/chem/6d5} is our required balance equation.
