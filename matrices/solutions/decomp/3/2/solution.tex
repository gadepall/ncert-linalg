
First we find orthogonal vectors $\vec{m_1}$ and $\vec{m_2}$ to the given normal vector $\vec{n}$. Let, $\vec{m}$ = $\myvec{a\\b\\c}$, then
\begin{align}
\vec{m^T}\vec{n} &= 0\\
\implies\myvec{a&b&c}\myvec{6\\-3\\2} &= 0\\
\implies6a-3b+2c &= 0\\
\intertext{Putting a=1 and b=0 we get,}
\vec{m_1} &= \myvec{1\\0\\3}\\
\intertext{Putting a=0 and b=1 we get,}
\vec{m_2} &= \myvec{0\\1\\\frac{3}{2}}
\end{align}
Now we solve the equation,
\begin{align}
\vec{M}\vec{x} &= \vec{b}\label{eq:solutions/3/2/eq1}\\
\intertext{Putting values in \eqref{eq:solutions/3/2/eq1},}
\myvec{1&0\\0&1\\3&\frac{3}{2}}\vec{x} &= \myvec{2\\5\\-3}\label{eq:solutions/3/2/eq2}
\end{align}
Now, to solve \eqref{eq:solutions/3/2/eq2}, we perform Singular Value Decomposition on $\vec{M}$ as follows,
\begin{align}
\vec{M}=\vec{U}\vec{S}\vec{V}^T\label{eq:solutions/3/2/eqSVD}
\end{align}
Where the columns of $\vec{V}$ are the eigen vectors of $\vec{M}^T\vec{M}$ ,the columns of $\vec{U}$ are the eigen vectors of $\vec{M}\vec{M}^T$ and $\vec{S}$ is diagonal matrix of singular value of eigenvalues of $\vec{M}^T\vec{M}$.
\begin{align}
\vec{M}^T\vec{M}=\myvec{10&\frac{9}{2}\\\frac{9}{2}&\frac{13}{4}}\label{eq:solutions/3/2/eqMTM}\\
\vec{M}\vec{M}^T=\myvec{1&0&3\\0&1&\frac{3}{2}\\3&\frac{3}{2}&\frac{45}{4}}
\end{align}
From \eqref{eq:solutions/3/2/eq1} putting \eqref{eq:solutions/3/2/eqSVD} we get,
\begin{align}
\vec{U}\vec{S}\vec{V}^T\vec{x} & = \vec{b}\\
\implies\vec{x} &= \vec{V}\vec{S_+}\vec{U^T}\vec{b}\label{eq:solutions/3/2/eqX}
\end{align}
Where $\vec{S_+}$ is Moore-Penrose Pseudo-Inverse of $\vec{S}$.Now, calculating eigen value of $\vec{M}\vec{M}^T$,
\begin{align}
\mydet{\vec{M}\vec{M}^T - \lambda\vec{I}} &= 0\\
\implies\myvec{1-\lambda&0&3\\0&1-\lambda&\frac{3}{2}\\3&\frac{3}{2}&\frac{45}{4}-\lambda} &=0\\
\implies\lambda^3-\frac{53}{4}\lambda^2+\frac{49}{4}\lambda &=0
\end{align}
Hence eigen values of $\vec{M}\vec{M}^T$ are,
\begin{align}
\lambda_1 &= \frac{49}{4}\\
\lambda_2 &= 1\\
\lambda_3 &= 0
\end{align}
Hence the eigen vectors of $\vec{M}\vec{M}^T$ are,
\begin{align}
\vec{u}_1=\myvec{\frac{4}{15}\\\frac{2}{15}\\1},
\vec{u}_2=\myvec{-\frac{1}{2}\\1\\0},
\vec{u}_3=\myvec{-3\\-\frac{3}{2}\\1}
\intertext{Normalizing the eigen vectors we get,}
\vec{u}_1=\myvec{\frac{4}{7\sqrt{5}}\\\frac{2}{7\sqrt{5}}\\\frac{3\sqrt{5}}{7}},
\vec{u}_2=\myvec{-\frac{1}{\sqrt{5}}\\\frac{2}{\sqrt{5}}\\0},
\vec{u}_3=\myvec{-\frac{6}{7}\\-\frac{3}{7}\\\frac{2}{7}}
\end{align}
Hence we obtain $\vec{U}$ of \eqref{eq:solutions/3/2/eqSVD} as follows,
\begin{align}
\vec{U}=\myvec{\frac{4}{7\sqrt{5}}&-\frac{1}{\sqrt{5}}&-\frac{6}{7}\\\frac{2}{7\sqrt{5}}&\frac{2}{\sqrt{5}}&-\frac{3}{7}\\\frac{3\sqrt{5}}{7}&0&\frac{2}{7}}\label{eq:solutions/3/2/eqU}
\end{align}
After computing the singular values from eigen values $\lambda_1, \lambda_2, \lambda_3$ we get $\vec{S}$ of \eqref{eq:solutions/3/2/eqSVD} as follows,
\begin{align}
\vec{S}=\myvec{\frac{7}{2}&0\\0&1\\0&0}\label{eq:solutions/3/2/eqS}
\end{align}
Now, calculating eigen value of $\vec{M}^T\vec{M}$,
\begin{align}
\mydet{\vec{M}^T\vec{M} - \lambda\vec{I}} &= 0\\
\implies\myvec{10-\lambda&\frac{9}{2}\\\frac{9}{2}&\frac{13}{4}-\lambda} &=0\\
\implies\lambda^2-\frac{53}{4}\lambda+\frac{49}{4} &=0
\end{align}
Hence eigen values of $\vec{M}^T\vec{M}$ are,
\begin{align}
\lambda_4 &= \frac{49}{4}\\
\lambda_5 &= 1
\end{align}
Hence the eigen vectors of $\vec{M}^T\vec{M}$ are,
\begin{align}
\vec{v}_1=\myvec{2\\1},
\vec{v}_2=\myvec{-\frac{1}{2}\\1}
\intertext{Normalizing the eigen vectors we get,}
\vec{v}_1=\myvec{\frac{2}{\sqrt{5}}\\\frac{1}{\sqrt{5}}},
\vec{v}_2=\myvec{-\frac{1}{\sqrt{5}}\\\frac{2}{\sqrt{5}}}
\end{align}
Hence we obtain $\vec{V}$ of \eqref{eq:solutions/3/2/eqSVD} as follows,
\begin{align}
\vec{V}=\myvec{\frac{2}{\sqrt{5}}&-\frac{1}{\sqrt{5}}\\\frac{1}{\sqrt{5}}&\frac{2}{\sqrt{5}}}
\end{align}
Finally from \eqref{eq:solutions/3/2/eqSVD} we get the Singualr Value Decomposition of $\vec{M}$ as follows,
\begin{align}
\vec{M} = \myvec{\frac{4}{7\sqrt{5}}&-\frac{1}{\sqrt{5}}&-\frac{6}{7}\\\frac{2}{7\sqrt{5}}&\frac{2}{\sqrt{5}}&-\frac{3}{7}\\\frac{3\sqrt{5}}{7}&0&\frac{2}{7}}\myvec{\frac{7}{2}&0\\0&1\\0&0}\myvec{\frac{2}{\sqrt{5}}&-\frac{1}{\sqrt{5}}\\\frac{1}{\sqrt{5}}&\frac{2}{\sqrt{5}}}^T
\end{align}
Now, Moore-Penrose Pseudo inverse of $\vec{S}$ is given by,
\begin{align}
\vec{S_+} = \myvec{\frac{2}{7}&0&0\\0&1&0}
\end{align}
From \eqref{eq:solutions/3/2/eqX} we get,
\begin{align}
\vec{U}^T\vec{b}&=\myvec{-\frac{27}{7\sqrt{5}}\\\frac{8}{7\sqrt{5}}\\-\frac{33}{7}}\\
\vec{S_+}\vec{U}^T\vec{b}&=\myvec{-\frac{54}{49\sqrt{5}}\\\frac{8}{7\sqrt{5}}}\\
\vec{x} = \vec{V}\vec{S_+}\vec{U}^T\vec{b} &= \myvec{-\frac{100}{49}\\\frac{146}{49}}\label{eq:solutions/3/2/eqXSol1}
\end{align}
Verifying the solution of \eqref{eq:solutions/3/2/eqXSol1} using,
\begin{align}
\vec{M}^T\vec{M}\vec{x} = \vec{M}^T\vec{b}\label{eq:solutions/3/2/eqVerify}
\end{align}
Evaluating the R.H.S in \eqref{eq:solutions/3/2/eqVerify} we get,
\begin{align}
\vec{M}^T\vec{M}\vec{x} &= \myvec{-7\\\frac{1}{2}}\\
\implies\myvec{10&\frac{9}{2}\\\frac{9}{2}&\frac{13}{4}}\vec{x} &= \myvec{-7\\\frac{1}{2}}\label{eq:solutions/3/2/eqMateq}
\end{align}
Solving the augmented matrix of \eqref{eq:solutions/3/2/eqMateq} we get,
\begin{align}
\myvec{10&\frac{9}{2}&-7\\\frac{9}{2}&\frac{13}{4}&\frac{1}{2}} &\xleftrightarrow{R_1=\frac{1}{10}R_1}\myvec{1&\frac{9}{20}&-\frac{7}{10}\\\frac{9}{2}&\frac{13}{4}&\frac{1}{2}}\\
&\xleftrightarrow{R_2=R_2-\frac{9}{2}R_1}\myvec{1&\frac{9}{20}&-\frac{7}{10}\\0&\frac{49}{40}&\frac{73}{20}}\\
&\xleftrightarrow{R_2=\frac{40}{49}R_2}\myvec{1&\frac{9}{20}&-\frac{7}{10}\\0&1&\frac{146}{49}}\\
&\xleftrightarrow{R_1=R_1-\frac{9}{20}R_1}\myvec{1&0&-\frac{100}{49}\\0&1&\frac{146}{49}}
\end{align}
Hence, Solution of \eqref{eq:solutions/3/2/eqVerify} is given by,
\begin{align}
\vec{x}=\myvec{-\frac{100}{49}\\\frac{146}{49}}\label{eq:solutions/3/2/eqX2}
\end{align}
Comparing results of $\vec{x}$ from \eqref{eq:solutions/3/2/eqXSol1} and \eqref{eq:solutions/3/2/eqX2} we conclude that the solution is verified.
