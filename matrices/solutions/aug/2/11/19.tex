\begin{enumerate}
    \item Given,
    \begin{align}
    \label{aug/2/19/eq:1}
        \vec{A}=\myvec{1 & -1 & 5\\ -1 & 2 & 1\\5 & 1 & 3}
    \end{align}
    Transposing the matrix,
    \begin{align}
        \label{aug/2/19/eq:2}
        \vec{A}^\top=\myvec{1 & -1 & 5 \\-1 & 2 & 1\\5 & 1 &3}
    \end{align}
    Using \eqref{aug/2/19/eq:1} and \eqref{aug/2/19/eq:2} we get,
    \begin{align}
        \vec{A}=\vec{A}^\top
    \end{align}
    %\begin{center}
        $\therefore \vec{A}$ is symmetric matrix. 
    %\end{center}
        
    \item Given,
    \begin{align}
    \label{aug/2/19/eq:4}
        \vec{A}=\myvec{0 & 1 & -1\\ -1 & 0 & 1\\1 & -1 & 0}
    \end{align}
    Transposing the matrix,
    \begin{align}
        \label{aug/2/19/eq:5}
        \vec{A}^\top=\myvec{0 & -1 & 1 \\1 & 0 & -1\\-1 & 1 & 0}
    \end{align}
    Using \eqref{aug/2/19/eq:4} and \eqref{aug/2/19/eq:5} we get,
    \begin{align}
        \vec{A}=-\vec{A}^\top
    \end{align}
    %\begin{center}
        $\therefore \vec{A}$ is skew symmetric matrix. 
    %\end{center}
    \end{enumerate}