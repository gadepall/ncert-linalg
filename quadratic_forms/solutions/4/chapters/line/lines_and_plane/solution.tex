\begin{enumerate}
\item put $\vec {x}  = \myvec{x\\0}$ in equation
\\ 
\begin{align}
\myvec{2 & 3}\myvec{x\\0} = \frac{187}{20}
\\
x= \frac{187}{40}
\end{align}
\\
put $\vec x \myvec{0\\y}$ in equation
\\
\begin{align}
\myvec{2 & 3}\myvec{0\\y} = \frac{187}{20}
\\
y= \frac{187}{60}
\\
\vec{P1} = \myvec{\frac{187}{40}\\0}, \vec{Q1} = \myvec{0\\\frac{187}{60}}
\end{align}



\item put $\vec x \myvec{x\\0}$ in equation 
\begin{align}
\myvec{1 & -\frac{1}{5}}\myvec{x\\0} = 10
\\
x= 10
\end{align}
\\
put $\vec x \myvec{0\\y}$ in equation
\\
\begin{align}
\myvec{1 & -\frac{1}{5}}\myvec{0\\y} = 10
\\
y= -50
\\
\vec{P2} = \myvec{10\\0}, \vec{Q2} = \myvec{0\\-50}
\end{align}


\item put $\vec x \myvec{x\\0}$ in equation 
\begin{align}
\myvec{-2 & 3}\myvec{x\\0} = 6
\\
x= -3
\end{align}
put $\vec x \myvec{0\\y}$ in equation
\\
\begin{align}
\myvec{-2 & 3}\myvec{0\\y} = 6
\\
y= 2
\\
\vec{P3} = \myvec{-3\\0}, \vec{Q3} = \myvec{0\\2}
\end{align}




\item there is no constant in the line equation thus it passes through the origin.
\\
put $\vec x \myvec{3\\y}$ in equation
\\
\begin{align}
\myvec{1 & -3}\myvec{3\\y} = 0
\\
y= 1
\\
\vec{P4} = \myvec{0\\0}, \vec{Q4} = \myvec{3\\1}
\end{align}


\item there is no constant in the line equation thus it passes through the origin
\\
put $\vec x \myvec{1\\y}$ in equation
\\
\begin{align}
\myvec{2 & -1}\myvec{1\\y} = 0
\\
y= 1
\\
\vec{P5} = \myvec{0\\0}, \vec{Q5} = \myvec{1\\2}
\end{align}


\item put $\vec x \myvec{x\\0}$ in equation
\begin{align}
\myvec{3 & 0}\myvec{x\\0} = -2
\\
x= -\frac{2}{3}
\end{align}
\\
we can see in this equation the  value of x coordinate does not depend on the y coordinate so we can say that it is parallel to the y-axis.





\item put $\vec x \myvec{x\\0}$ in equation 
\begin{align}
\myvec{0& 1}\myvec{0\\y} = 2
\\
y= 2
\end{align}
\\
we can see in this equation the  value of y coordinate does not depend on the x coordinate so we can say that it is parallel to the x-axis.



\item put $\vec x \myvec{x\\0}$ in equation 
\begin{align}
\myvec{2 & 0}\myvec{x\\0} = 5
\\
x= -\frac{5}{2}
\end{align}
\\
we can see in this equation the  value of x coordinate does not depend on the y coordinate so we can say that it is parallel to the y-axis.
\begin{figure}[!ht]
	\centering
	\includegraphics[width=\columnwidth]{./solutions/4/figures/line/lines_and_planes/lines_and_planes.eps}
	\caption{lines }
	\label{fig:3.7.4_lines}
\end{figure}
\begin{lstlisting}
solutions/4/codes/line/lines_and_planes/plane_and_line.py
\end{lstlisting} 
\end{enumerate}
