\begin{figure}[h!]
\centering
\resizebox{\columnwidth}{!}
    {
    
\tikzset{every picture/.style={line width=0.75pt}} %set default line width to 0.75pt        

\begin{tikzpicture}[x=0.75pt,y=0.75pt,yscale=-1,xscale=1]

\draw   (335.1,58.89) -- (594.49,270.28) -- (334.1,663.59) -- (334.1,663.59) -- (73.72,270.28) -- cycle ;

\draw    (73.72,270.28) -- (594.49,270.28) ;

\draw    (335.1,58.89) -- (334.1,663.59) ;

\draw    (196.49,158.07) -- (209.49,176.07) ;

\draw    (465.49,152.07) -- (453.49,170.07) ;

\draw    (486.49,417.07) -- (502.61,431.11) ;

\draw    (481.49,425.07) -- (496.61,439.11) ;
 
\draw    (190.55,426.14) -- (174.55,440.14) ;

\draw    (195.55,434.14) -- (179.55,448.14) ;

\draw (43.08,247.59) node [anchor=north west][inner sep=0.75pt]   [align=left] {{\LARGE A}};

\draw (601.08,254.59) node [anchor=north west][inner sep=0.75pt]   [align=left] {{\LARGE B}};

\draw (326.08,25.59) node [anchor=north west][inner sep=0.75pt]   [align=left] {{\LARGE P}};

\draw (341.1,278.28) node [anchor=north west][inner sep=0.75pt]   [align=left] {{\LARGE M}};

\draw (345.08,657.36) node [anchor=north west][inner sep=0.75pt]   [align=left] {{\LARGE Q}};

\end{tikzpicture}

    }
\end{figure}
In order to prove that line $PQ $ is the perpendicular bisector of $AB$, two conditions need to be met:
\begin{enumerate}
        \item $PQ \perp AB $
    \item $AM$=$BM$
\end{enumerate}
These conditions can be proved as follow:
\noindent It is given that the points $\vec{P}$ and $\vec{Q}$  are equidistant from the points $\vec{A}$ and $\vec{B}$. Thus we can write:
\begin{align}
    \norm{\vec{P}-\vec{A}}=\norm{\vec{P}-\vec{B}} \label{eq:solutions/1/9/eq2.2}\\
    \norm{\vec{Q}-\vec{A}}=\norm{\vec{Q}-\vec{B}} \label{eq:solutions/1/9/eq2.3}
\end{align}
Squaring both sides of equations \ref{eq:solutions/1/9/eq2.2} and expanding further, we can write:
\begin{align}
    \brak{\vec{P}-\vec{A}}^T\brak{\vec{P}-\vec{A}}=\brak{\vec{P}-\vec{B}}^T\brak{\vec{P}-\vec{B}} \\
    \vec{P}^T\vec{P}-\vec{P}^T\vec{A}-\vec{A}^T\vec{P}+\vec{A}^T\vec{A}= \nonumber \\ 
     \vec{P}^T\vec{P}-\vec{P}^T\vec{B}-\vec{B}^T\vec{P}+\vec{B}^T\vec{B} \\
     \therefore \vec{A}^T\vec{A}-\vec{B}^T\vec{B}=-2\vec{P}^T\vec{B}+2\vec{P}^T\vec{A}\label{eq:solutions/1/9/eq3.5}
\end{align}
Similarly, Squaring both sides of equations \ref{eq:solutions/1/9/eq2.3} and expanding further, we can write:
\begin{align}
    \brak{\vec{Q}-\vec{A}}^T\brak{\vec{Q}-\vec{A}}=\brak{\vec{Q}-\vec{B}}^T\brak{\vec{Q}-\vec{B}} \\
    \vec{Q}^T\vec{Q}-\vec{Q}^T\vec{A}-\vec{A}^T\vec{Q}+\vec{A}^T\vec{A}= \nonumber \\ 
     \vec{Q}^T\vec{Q}-\vec{Q}^T\vec{B}-\vec{B}^T\vec{Q}+\vec{B}^T\vec{B} \\
     \therefore \vec{A}^T\vec{A}-\vec{B}^T\vec{B}=-2\vec{Q}^T\vec{B}+2\vec{Q}^T\vec{A}\label{eq:solutions/1/9/eq3.8}
\end{align}
From equations \ref{eq:solutions/1/9/eq3.5} and \ref{eq:solutions/1/9/eq3.8}, we can write:
\begin{align}
    2\vec{P}^T\brak{\vec{A}-\vec{B}}=2\vec{Q}^T\brak{\vec{A}-\vec{B}}\\
    \vec{P}^T\brak{\vec{A}-\vec{B}}-\vec{Q}^T\brak{\vec{A}-\vec{B}}=0\\
    \brak{\vec{P}-\vec{Q}}^T\brak{\vec{A}-\vec{B}}=0
\end{align}
Thus, Segment $PQ$ is perpendicular to segment $AB$ ($PQ \perp AB$).\\ \\
From the figure, equations \ref{eq:solutions/1/9/eq2.2} can also be written as:
\begin{align}
    \norm{\brak{\vec{P}-\vec{M}}+\brak{\vec{M}-\vec{A}}}=\norm{\brak{\vec{P}-\vec{M}}+\brak{\vec{M}-\vec{B}}}
\end{align}
Squaring and expanding both the sides, we get:
\begin{align}
\brak{\brak{\vec{P}-\vec{M}}+\brak{\vec{M}-\vec{A}}}^T\brak{\brak{\vec{P}-\vec{M}}+\brak{\vec{M}-\vec{A}}} = \nonumber \\
\brak{\brak{\vec{P}-\vec{M}}+\brak{\vec{M}-\vec{B}}}^T\brak{\brak{\vec{P}-\vec{M}}+\brak{\vec{M}-\vec{B}}}
\end{align}
\begin{align}
\brak{\vec{P}-\vec{M}}^T\brak{\vec{P}-\vec{M}}+\brak{\vec{P}-\vec{M}}^T\brak{\vec{M}-\vec{A}}+ \nonumber \\\brak{\vec{M}-\vec{A}}^T\brak{\vec{P}-\vec{M}}+\brak{\vec{M}-\vec{A}}^T\brak{\vec{M}-\vec{A}} = \nonumber \\
\brak{\vec{P}-\vec{M}}^T\brak{\vec{P}-\vec{M}}+\brak{\vec{P}-\vec{M}}^T\brak{\vec{M}-\vec{B}}+ \nonumber \\\brak{\vec{M}-\vec{B}}^T\brak{\vec{P}-\vec{M}}+\brak{\vec{M}-\vec{B}}^T\brak{\vec{M}-\vec{B}}\\
\norm{\brak{\vec{M}-\vec{A}}}^2+2\brak{\vec{M}-\vec{A}}^T\brak{\vec{P}-\vec{M}} = \nonumber \\
\norm{\brak{\vec{M}-\vec{B}}}^2+2\brak{\vec{M}-\vec{B}}^T\brak{\vec{P}-\vec{M}} \label{eq:solutions/1/9/eq3.15}
\end{align}
Since, $PQ\perp AB$. Hence, we can write:
\begin{align}
\brak{\vec{M}-\vec{A}}^T\brak{\vec{P}-\vec{M}}=\brak{\vec{M}-\vec{B}}^T\brak{\vec{P}-\vec{M}}=0 \label{eq:solutions/1/9/eq3.16}
\end{align}
From equation \ref{eq:solutions/1/9/eq3.15} and \ref{eq:solutions/1/9/eq3.16}, we get:
\begin{align}
    \norm{\brak{\vec{M}-\vec{A}}}=\norm{\brak{\vec{M}-\vec{B}}}
\end{align}
Thus, $M$ is the midpoint of segment $AB$ ($AM=BM$).
Thus, Segment $PQ$ is perpendicular bisector of segment $AB$.
