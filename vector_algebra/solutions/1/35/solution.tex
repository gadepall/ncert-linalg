\begin{figure}[hb]
	\centering
	\centering
	\resizebox{\columnwidth}{!}{\begin{tikzpicture} 
        \coordinate (B) at (2.5, -2.5) {};
        \coordinate (A) at (0, 0) {};
        \coordinate (P) at (5, 0) {};
        \coordinate (C) at (2.5, 2.5) {};

        \draw (C)node[above]{$C$}--(A)node[below]{$A$}--(P)node[above]{$P$}--cycle;
        \draw (P)node[above]{$P$}--(A)node[below]{$A$}--(B)node[below]{$B$}--cycle;
\tkzMarkRightAngle[size=.2](P,C,A);
\tkzLabelAngle[dist=.5](P,A,B){};
\tkzMarkRightAngle[size=.2](P,B,A);
\tkzLabelAngle[dist=.5](P,A,C){};
\end{tikzpicture}
}
	\caption{$\triangle ABC \text{ and } \triangle PQR$}
	\label{eq:solutions/1/35/myfig}
\end{figure}
Given Condition in the question are:
\begin{align}
&AB = PQ\label{eq:solutions/1/35/eq1} \\
&BC = QR\label{eq:solutions/1/35/eq2} \\
&AM = PN\label{eq:solutions/1/35/eq3}
\end{align}
As M and N are medians of triangle ABC and triangle PQR respectively, we deduce the following:
\begin{align}
&\vec{M} = \frac{\vec{B}+\vec{C}}{2} \\
&\implies 2\vec{M} = \brak{\vec{B+C}}\\
&\implies\brak{\vec{B-M}} = \brak{\vec{M-C}}\\
&\implies\norm{\vec{B}-\vec{M}}=\norm{\vec{M}-\vec{C}}\label{eq:solutions/1/35/eq5} \\
\text{Also}\\
&\vec{N} = \frac{\vec{Q}+\vec{R}}{2}\label{eq:solutions/1/35/eq6} \\
&\implies2\vec{N} = \brak{\vec{Q+R}}\\
&\implies\brak{\vec{Q-N}} = \brak{\vec{N-R}}\\
&\implies\norm{\vec{Q}-\vec{N}}=\norm{\vec{N}-\vec{R}}\label{eq:solutions/1/35/eq7}
\end{align}
Refer\eqref{eq:solutions/1/35/eq5}and \eqref{eq:solutions/1/35/eq7}
\begin{align}
&\norm{\vec{B}-\vec{M}}= \norm{\vec{Q}-\vec{N}}
\end{align}
Hence in triangle ABM and triangle PQN sides AB,BM and MA are equal to PQ,QN and NP so by SSS congruency criteria 
\begin{align}
    \triangle{ABM}\cong\triangle{PQN}\label{eq:solutions/1/35/eqA}
\end{align}
Now for proving congruence of triangle ABC and triangle PQR we know that the corresponding angles of congurent triangles are equal and to prove that we make a hypothesis and proceed as follows
\begin{align}
\angle ABM = \angle PQN\label{eq:solutions/1/35/eq8} 
\end{align} 
and one of the proved condition
\begin{align}
  BM=QN\label{eq:solutions/1/35/eq9}  
\end{align}
Refer\eqref{eq:solutions/1/35/eq8}\\
\begin{align}
\cos \angle ABM=\cos \angle PQN    
\end{align}
\begin{align}
\frac{\left( \vec{ B - A} \right)^T  \left( \vec{B - M } \right)}{\norm{\vec{ B - A}} \norm{\vec{B - M}}}=\frac{\left( \vec{ Q - P} \right)^T  \left( \vec{Q - N} \right)}{\norm{\vec{ Q - P}} \norm{\vec{Q - N}}}
\end{align}
Equating \eqref{eq:solutions/1/35/eq9}
\begin{multline}
\label{eq:solutions/1/35/eq10} 
\implies\frac{\left( \vec{ B - A} \right)^T  \left( \vec{B - M} \right)}{\norm{\vec{ B - A}}}=\\
\frac{\left( \vec{ Q - P} \right)^T  \left( \vec{Q - N } \right)}{\norm{\vec{ Q - P}}}
\end{multline}
It can be shown that
\begin{multline}
\label{eq:solutions/1/35/eq11}
 \left( \vec{ B - A} \right)^T  \left( \vec{B - M} \right)=\\
 \norm{\vec{A - B}}^2 - \left ( \vec{  A - M  }\right)^T \left( \vec{A - B} \right)
 \end{multline}
 \begin{multline}
 \label{eq:solutions/1/35/eq12}
 \left( \vec{ Q - P} \right)^T  \left( \vec{ Q- N } \right)=\\
 \norm{\vec{P - Q}}^2 - \left ( \vec{  P - N  }\right)^T \left( \vec{P - Q} \right)
\end{multline}
Substituting \eqref{eq:solutions/1/35/eq11} and \eqref{eq:solutions/1/35/eq12} in \eqref{eq:solutions/1/35/eq10}
\begin{multline}
\norm{\vec{A - B}}- \frac{\left( \vec{ A - M} \right)^T \left( \vec{A - B} \right)}{\norm{\vec{ B - A}}} =\\ \norm{\vec{P - Q}}- \frac{\left( \vec{ P - N} \right)^T \left( \vec{P- Q} \right)}{\norm{\vec{ Q - P}}}
\end{multline}
\begin{multline}
\norm{\vec{A - B}}- \norm{\vec{A - M}}\cos\angle BAM =\\  \norm{\vec{P - Q}}- \norm{\vec{P - N}}\cos\angle QPN   
\end{multline}
 Refer \eqref{eq:solutions/1/35/eq1} and \eqref{eq:solutions/1/35/eq3}
\begin{align}
\norm{\vec{A - B}}= \norm{\vec{P - Q}}
\end{align}
\begin{align}
\norm{\vec{A - M}}= \norm{\vec{P - N}}
\end{align}
\begin{align}
\therefore \cos\angle BAM =  \cos\angle QPN
\end{align}
\begin{align}
\implies \angle BAM = \angle QPN
\end{align}
Hence our hypothesis is right as we prove that corresponding angles of congurent triangles are equal. So we get 
\begin{align}
\angle ABM=\angle PQN    
\end{align}
\begin{align}
\therefore \angle ABC=\angle PQR    
\end{align}
So by applying SAS criteria we conclude that\\
\begin{align}
\triangle{ABC}\cong\triangle{PQR}
\end{align}
