In the given problem,
\begin{align}
	\vec{A}_1 = \myvec{1 \\ 2 \\ 3}, \vec{m}_1 = \myvec{-3\\ 2k\\ 2}, \vec{A}_2 = \myvec{3\\ 1 \\ 6}, \vec{m}_2 = \myvec{3k \\ 1 \\ -5}
\end{align}
To find the value of k, let's assume that the given lines are perpendicular to each other. Then the dot product of their direction vectors should be 0. i.e.,
\begin{align}
	\vec{m}_1 \vec{m}_2 &= 0\\
	\implies \myvec{-3\\ 2k\\ 2} \myvec{3k \\ 1 \\ -5} &= 0\\
	\implies k &= -\frac{10}{7}
\end{align}
The lines will intersect if 
\begin{align}
\vec{A}_1 + \lambda_1 \vec{m}_1 &= \vec{A}_2 + \lambda_2 \vec{m}_2 \\
\implies \myvec{1 \\ 2 \\ 3} + \lambda_1\myvec{-3\\ 2k\\ 2} &= \myvec{3\\ 1 \\ 6} + \lambda_2\myvec{3k \\ 1 \\ -5}\\
\implies \lambda_1\myvec{-3\\ 2k\\ 2} - \lambda_2\myvec{3k \\ 1 \\ -5} &= \myvec{3\\ 1 \\ 6}  - \myvec{1 \\ 2 \\ 3}\\
\implies \myvec{-3 & 3k \\ 2k & 1\\ 2 & -5} \myvec{\lambda_1 \\ \lambda_2} &= \myvec{2 \\ -1 \\ 3}\\
\implies \myvec{-3 & -\frac{30}{7} \\ -\frac{20}{7} & 1\\ 2 & -5} \myvec{\lambda_1 \\ \lambda_2} &= \myvec{2 \\ -1 \\ 3} \label{eq:solutions/3/9/eql1l2}
\end{align}
Row reducing the augmented matrix, 
\begin{align}
	\myvec{-3 & -\frac{30}{7} & 2\\ -\frac{20}{7} & 1 & -1\\ 2 & -5 & 3} \xleftrightarrow[R_2 \leftarrow R_2 + \frac{20}{7}R_1]{R_1 \leftarrow -\frac{R_1}{3}} \myvec{1 & \frac{10}{7} & -\frac{2}{3}\\ 0 & \frac{249}{49} & -\frac{61}{21}\\ 2 & -5 & 3}\\
	\xleftrightarrow[R_2 \leftarrow \frac{49}{249}R_2]{R_3 \leftarrow R_3 - 2R_1} \myvec{1 & \frac{10}{7} & -\frac{2}{3}\\ 0 & \frac{249}{49} & -\frac{61}{21}\\ 2 & -5 & 3}\\
	\xleftrightarrow[R_2 \leftarrow \frac{49}{249}R_2]{R_3 \leftarrow R_3 - 2R_1} \myvec{1 & \frac{10}{7} & -\frac{2}{3}\\ 0 & 1 & -\frac{427}{747}\\ 0 & -\frac{55}{7} & \frac{13}{3}}\\
	\xleftrightarrow[R_3 \leftarrow -\frac{747}{118}R_3]{R_3 \leftarrow R_3 +\frac{55}{7}R_2} \myvec{1 & \frac{10}{7} & -\frac{2}{3}\\ 0 & 1 & -\frac{427}{747}\\ 0 & 0& 1}\\
	\xleftrightarrow[R_1 \leftarrow R_1 + \frac{2}{3}R_3 - \frac{10}{7}R_2]{R_2 \leftarrow R_2 +\frac{427}{47}R_3} \myvec{1 & 0 & 0\\ 0 & 1 & 0\\ 0 & 0& 1}
\end{align}
The above matrix has $rank = 3$. Hence, the lines do not intersect which implies that the given lines are skew lines. To find the closest points using SVD, consider the equation \eqref{eq:solutions/3/9/eql1l2} which can be expressed as 
\begin{align}
	\vec{M}\vec{x}&=\vec{b}\label{eq:solutions/3/9/mx=b}
\end{align}
By singular value decomposition $\vec{M}$
can be expressed as 
\begin{align}
	\vec{M}&=\vec{U}\vec{S}\vec{V}^T\label{eq:solutions/3/9/main}
\end{align}
where the columns of $\vec{V}$ are the eigenvectors of $\vec{M}^T\vec{M}$, the columns of $\vec{U}$ are the eigenvectors of $\vec{M}\vec{M}^T$ and $\vec{S}$ is diagonal matrix of singular value of eigenvalues of $\vec{M}^T\vec{M}$.
\begin{align}
	\vec{M}^T\vec{M} &=\myvec{\frac{1037}{49} & 0\\ 0 & \frac{2174}{49}}\\
	\vec{M}\vec{M}^T &=\myvec{\frac{1341}{49} & \frac{30}{7} & \frac{108}{7}\\ \frac{30}{7} & \frac{449}{49} & -\frac{75}{7} \\ \frac{108}{7} & -\frac{75}{7} & 29}
\end{align}
{To get $\vec{V}$ and $\vec{S}$ }
The characteristic equation of $\vec{M}^T\vec{M}$ is obtained by evaluating the determinant 
\begin{align}
    \mydet{ \frac{1037}{49} - \lambda & 0\\ 0 & \frac{2174}{49} - \lambda} &= 0\\
	\implies \lambda^2 - \frac{286699}{637}\lambda + \sbrak{\frac{1037 \times 2174}{49^2}} &=0\label{eq:solutions/3/9/eqroots}
\end{align}
The eigenvalues are the roots of equation \ref{eq:solutions/3/9/eqroots} is given by 
\begin{align}
	\lambda_{11}&= \frac{2174}{49} \label{eq:solutions/3/9/eqeig1}\\
	\lambda_{12}&=\frac{1037}{49} \label{eq:solutions/3/9/eqeig2}
\end{align}
The corresponding eigen vectors are, 
\begin{align}
	\vec{u}_{11} &=\myvec{0 \\ 1}\\
	\vec{u}_{12} &=\myvec{1 \\ 0}
\end{align}
\begin{align}
\therefore	\vec{V} &=\myvec{0 & 1\\1 & 0}
\end{align}
$\vec{S}$ is given by 
\begin{align}
	\vec{S}&=\myvec{\frac{\sqrt{2174}}{7} & 0\\ 0 & \frac{\sqrt{1037}}{7} \\ 0 &0}
\end{align}
{To get $\vec{U}$ }
The characteristic equation of $\vec{M}\vec{M}^T$ is obtained by evaluating the determinant 
\begin{align}
    \mydet{\frac{1341}{49} - \lambda & \frac{30}{7} & \frac{108}{7}\\ \frac{30}{7} & \frac{449}{49} - \lambda & -\frac{75}{7} \\ \frac{108}{7} & -\frac{75}{7} & 29 - \lambda} &= 0\\
	\implies -\lambda^3 + \frac{3211}{49}\lambda^2 - \frac{2254438}{2401} \lambda &=0 \label{eq:solutions/3/9/equroots}
\end{align}
The eigenvalues are the roots of equation \ref{eq:solutions/3/9/equroots} is given by 
\begin{align}
	\lambda_{21}&= \frac{2174}{49} 
%\label{eq:solutions/3/9/eqeig1}
\\
	\lambda_{22}&= \frac{1037}{49}
%\label{eq:solutions/3/9/eqeig2}
\\
	\lambda_{23}&=0
\end{align}
The corresponding eigen vectors are , 
\begin{align}
	\vec{u}_{21}=\myvec{-\frac{6}{7}\\ \frac{1}{5}\\ -1},
	\vec{u}_{22}=\myvec{-\frac{3}{2}\\ -\frac{10}{7}\\1},
	\vec{u}_{23}=\myvec{-\frac{602}{747} \\ \frac{384}{249} \\ 1}
\end{align}
Normalizing the eigen vectors, 
\begin{align}
	\norm{\vec{u}_{21}} &=\sqrt{\left(\frac{-6}{7}\right)^2+\left(\frac{1}{5}\right)^2+1} =\frac{\sqrt{2174}}{35}\\
	\implies \vec{u}_{21} &=\myvec{-\frac{210}{7 \sqrt{2174}} \\ \frac{35}{5 \sqrt{2176}} \\ -\frac{35}{\sqrt{2174}}}
\end{align}
\begin{align}
	\norm{\vec{u}_{22}}&=\sqrt{\left(\frac{-3}{2}\right)^2+\left(\frac{-10}{7}\right)^2+1}=\frac{\sqrt{1037}}{14}\\
	\implies \vec{u}_{22}&=\myvec{-\frac{42}{2\sqrt{1037}} \\ -\frac{20}{\sqrt{1037}} \\ \frac{14}{\sqrt{1037}}}
\end{align}
\begin{align}
	\norm{\vec{u}_{23}} &=\sqrt{\left(\frac{-602}{747}\right)^2+\left(\frac{384}{249}\right)^2+1}= \frac{\sqrt{4027743}}{1000}\\
	\implies \vec{u}_{23} &=\myvec{-\frac{602000}{747\sqrt{4027743}} \\ \frac{384000}{249\sqrt{4027743}}\\
		                      \frac{1000}{\sqrt{4027743}}}
\end{align}
\begin{align}
	\vec{U}=\myvec{\frac{-210}{7\sqrt{2174}} & \frac{-42}{2\sqrt{1037}} & \frac{-602000}{747\sqrt{4027743}} &\\ \frac{35}{5\sqrt{2174}}& \frac{-20}{\sqrt{1037}}&  \frac{384000}{249\sqrt{4027743}} \\
		\frac{-35}{\sqrt{2174}}& \frac{14}{\sqrt{1037}} &   \frac{1000}{\sqrt{4027743}}}
\end{align}
{To get $\vec{x}$ }
Using \eqref{eq:solutions/3/9/main} we rewrite $\vec{M}$ as follows,
\begin{multline}
		\myvec{-3&-\frac{30}{7} \\-\frac{20}{7} & 1 \\ 2 & -5} =
		\myvec{\frac{-210}{7\sqrt{2174}} & \frac{-42}{2\sqrt{1037}} & \frac{-602000}{747\sqrt{4027743}} &\\ \frac{35}{5\sqrt{2174}}& \frac{-20}{\sqrt{1037}}&  \frac{384000}{249\sqrt{4027743}} \\
			\frac{-35}{\sqrt{2174}}& \frac{14}{\sqrt{1037}} &   \frac{1000}{\sqrt{4027743}}} \\
    	\myvec{\frac{\sqrt{2174}}{7} & 0\\ 0 & \frac{\sqrt{1037}}{7} \\ 0 &0}
		\myvec{0 & 1\\1 & 0}^T
\end{multline}
By substituting the equation \eqref{eq:solutions/3/9/main} in equation \eqref{eq:solutions/3/9/mx=b} we get 
\begin{align}
	\vec{U}\vec{S}\vec{V}^T\vec{x} & = \vec{b}\\
	\implies\vec{x} &= \vec{V}\vec{S}_+\vec{U^T}\vec{b} \label{eq:solutions/3/9/eqX}
\end{align}
where $\vec{S}_+$ is Moore-Penrose Pseudo-Inverse of $\vec{S}$
\begin{align}
	\vec{S}_+=\myvec{\frac{7}{\sqrt{2174}} & 0 & 0 \\ 0 & \frac{7}{\sqrt{1037}} & 0}
\end{align}
From \eqref{eq:solutions/3/9/eqX} we get,
\begin{align}
	\vec{U}^T\vec{b}&=\myvec{\frac{-172}{\sqrt{2174}} \\ \frac{20}{\sqrt{1037}} \\ \frac{-115000}{747\sqrt{4027743}}}\\ 
	\vec{S}_+\vec{U}^T\vec{b}&=\myvec{\frac{-602}{1087} \\ \frac{140}{1037}}\\
	\vec{x} = \vec{V}\vec{S}_+\vec{U}^T\vec{b} &= \myvec{\frac{140}{1037} \\ \frac{-602}{1087}}\label{eq:solutions/3/9/eqXSol1}
\end{align}
{Verification of $\vec{x}$}
Verifying the solution of \eqref{eq:solutions/3/9/eqXSol1} using,
\begin{align}
	\vec{M}^T\vec{M}\vec{x} = \vec{M}^T\vec{b}\label{eq:solutions/3/9/eqVerify}
\end{align}
Evaluating the R.H.S in \eqref{eq:solutions/3/9/eqVerify} we get,
\begin{align}
	\vec{M}^T\vec{M}\vec{x} &= \myvec{\frac{20}{7}\\ - \frac{172}{7}}\\
	\implies\myvec{\frac{1037}{49} & 0 \\ 0 & \frac{2174}{49}}\vec{x} &= \myvec{\frac{20}{7}\\ -\frac{172}{7}}\label{eq:solutions/3/9/eqMateq}
\end{align}
Solving the augmented matrix of \eqref{eq:solutions/3/9/eqMateq} we get,
\begin{align}
	\myvec{\frac{1037}{49} & 0 & \frac{20}{7} \\ 0 & \frac{2174}{49} & -\frac{172}{7}} 
	\xleftrightarrow[R_2 \leftarrow \frac{49}{2174}R_2]{R_1 \leftarrow \frac{49}{1037}R_1}
	\myvec{1 & 0 & \frac{140}{1037} \\ 0 & 1 & -\frac{602}{1087}}
\end{align}
Hence, Solution of \eqref{eq:solutions/3/9/eqVerify} is given by,
\begin{align}
	\vec{x}=\myvec{\frac{140}{1037} \\ \frac{-602}{1087}}\label{eq:solutions/3/9/eqX2}
\end{align}
Comparing results of $\vec{x}$ from \eqref{eq:solutions/3/9/eqXSol1} and \eqref{eq:solutions/3/9/eqX2} we conclude that the solution is verified.
