\documentclass[journal,12pt,twocolumn]{IEEEtran}

\usepackage{setspace}
\usepackage{gensymb}


\singlespacing

\usepackage[cmex10]{amsmath}
%\usepackage{amsthm}
%\interdisplaylinepenalty=2500
%\savesymbol{iint}
%\usepackage{txfonts}
%\restoresymbol{TXF}{iint}
%\usepackage{wasysym}
\usepackage{amsthm}

\usepackage{mathrsfs}
\usepackage{txfonts}
\usepackage{stfloats}
\usepackage{bm}
\usepackage{cite}
\usepackage{cases}
\usepackage{subfig}

\usepackage{longtable}
\usepackage{multirow}

\usepackage{enumitem}
\usepackage{mathtools}
\usepackage{steinmetz}
\usepackage{tikz}
\usepackage{circuitikz}
\usepackage{verbatim}
\usepackage{tfrupee}
\usepackage[breaklinks=true]{hyperref}

\usepackage{tkz-euclide} %loads TikZ and tkz-base

\usetikzlibrary{calc,math}
\usepackage{listings}
    \usepackage{color}                                          
    \usepackage{array}                                          
    \usepackage{longtable}                                      
    \usepackage{calc}                                           
    \usepackage{multirow}                                       
    \usepackage{hhline}                                         
    \usepackage{ifthen}
    \usepackage{lscape}     
\usepackage{multicol}
\usepackage{chngcntr}

\DeclareMathOperator*{\Res}{Res}

\renewcommand\thesection{\arabic{section}}
\renewcommand\thesubsection{\thesection.\arabic{subsection}}
\renewcommand\thesubsubsection{\thesubsection.\arabic{subsubsection}}

\renewcommand\thesectiondis{\arabic{section}}
\renewcommand\thesubsectiondis{\thesectiondis.\arabic{subsection}}
\renewcommand\thesubsubsectiondis{\thesubsectiondis.\arabic{subsubsection}}

\hyphenation{op-tical net-works semi-conduc-tor}
\def\inputGnumericTable{}                                 %%

\lstset{
%language=C,
frame=single, 
breaklines=true,
columns=fullflexible
}

\begin{document}

\newtheorem{theorem}{Theorem}[section]
\newtheorem{problem}{Problem}
\newtheorem{proposition}{Proposition}[section]
\newtheorem{lemma}{Lemma}[section]
\newtheorem{corollary}[theorem]{Corollary}
\newtheorem{example}{Example}[section]
\newtheorem{definition}[problem]{Definition}

\newcommand{\BEQA}{\begin{eqnarray}}
\newcommand{\EEQA}{\end{eqnarray}}
\newcommand{\define}{\stackrel{\triangle}{=}}
\bibliographystyle{IEEEtran}
\providecommand{\mbf}{\mathbf}
\providecommand{\pr}[1]{\ensuremath{\Pr\left(#1\right)}}
\providecommand{\qfunc}[1]{\ensuremath{Q\left(#1\right)}}
\providecommand{\sbrak}[1]{\ensuremath{{}\left[#1\right]}}
\providecommand{\lsbrak}[1]{\ensuremath{{}\left[#1\right.}}
\providecommand{\rsbrak}[1]{\ensuremath{{}\left.#1\right]}}
\providecommand{\brak}[1]{\ensuremath{\left(#1\right)}}
\providecommand{\lbrak}[1]{\ensuremath{\left(#1\right.}}
\providecommand{\rbrak}[1]{\ensuremath{\left.#1\right)}}
\providecommand{\cbrak}[1]{\ensuremath{\left\{#1\right\}}}
\providecommand{\lcbrak}[1]{\ensuremath{\left\{#1\right.}}
\providecommand{\rcbrak}[1]{\ensuremath{\left.#1\right\}}}
\theoremstyle{remark}
\newtheorem{rem}{Remark}
\newcommand{\sgn}{\mathop{\mathrm{sgn}}}
\providecommand{\abs}[1]{\left\vert#1\right\vert}
\providecommand{\res}[1]{\Res\displaylimits_{#1}} 
\providecommand{\norm}[1]{\left\lVert#1\right\rVert}
%\providecommand{\norm}[1]{\lVert#1\rVert}
\providecommand{\mtx}[1]{\mathbf{#1}}
\providecommand{\mean}[1]{E\left[ #1 \right]}
\providecommand{\fourier}{\overset{\mathcal{F}}{ \rightleftharpoons}}
%\providecommand{\hilbert}{\overset{\mathcal{H}}{ \rightleftharpoons}}
\providecommand{\system}{\overset{\mathcal{H}}{ \longleftrightarrow}}
	%\newcommand{\solution}[2]{\textbf{Solution:}{#1}}
\newcommand{\solution}{\noindent \textbf{Solution: }}
\newcommand{\cosec}{\,\text{cosec}\,}
\providecommand{\dec}[2]{\ensuremath{\overset{#1}{\underset{#2}{\gtrless}}}}
\newcommand{\myvec}[1]{\ensuremath{\begin{pmatrix}#1\end{pmatrix}}}
\newcommand{\mydet}[1]{\ensuremath{\begin{vmatrix}#1\end{vmatrix}}}
\numberwithin{equation}{subsection}
\makeatletter
\@addtoreset{figure}{problem}
\makeatother
\let\StandardTheFigure\thefigure
\let\vec\mathbf
\renewcommand{\thefigure}{\theproblem}
\def\putbox#1#2#3{\makebox[0in][l]{\makebox[#1][l]{}\raisebox{\baselineskip}[0in][0in]{\raisebox{#2}[0in][0in]{#3}}}}
     \def\rightbox#1{\makebox[0in][r]{#1}}
     \def\centbox#1{\makebox[0in]{#1}}
     \def\topbox#1{\raisebox{-\baselineskip}[0in][0in]{#1}}
     \def\midbox#1{\raisebox{-0.5\baselineskip}[0in][0in]{#1}}
\vspace{3cm}
\title{Matrix Theory (EE5609) Challenging Problem 3}
\author{Arkadipta De\\MTech Artificial Intelligence\\AI20MTECH14002}
\maketitle
\newpage
%\tableofcontents
\bigskip
\renewcommand{\thefigure}{\theenumi}
\renewcommand{\thetable}{\theenumi}
\begin{abstract}
This document proves the Cayley-Hamilton theorem.
\end{abstract}
Download latex codes from 
%
\begin{lstlisting}
https://github.com/Arko98/EE5609/tree/master/Challenge_2
\end{lstlisting}
%
\section{Problem}
Prove Cayley-Hamilton Theorem.
\section{Theorem Statement}
Every Square matrix satisfies its own characteristic equation.\\
Let, $\vec{A}$ be a square matrix of order $n$ and $p(\lambda) = \det({\vec{A}-\lambda \vec{I}})$ be the characteristic equation of \vec{A} in $\lambda$ and $\vec{I}$ is the identity matrix of order $n$ which is the same order of the matrix $\vec{A}$ then the charcteristic equation of $\vec{A}$ is given by,
\begin{align}  
\det({\vec{A}-\lambda \vec{I}}) &= 0\\
\implies\mydet{a_{11}-\lambda & a_{12} & a_{13} & \cdots & a_{nn}\\ 
a_{21} & a_{22}-\lambda & a_{23}& \cdots & a_{2n} \\
\vdots & \vdots & \vdots & \ddots &\vdots \\
a_{n1} & a_{n2} & a_{n3} & \cdots & a_{nn}-\lambda \\ } &= 0\\
\implies a_0 + a_1\lambda + a_2\lambda^{2} +  \cdots+a_n\lambda^{n} &= 0\label{eq1}
\end{align}
From Cayley-Hamilton theorem, the matrix $\vec{A}$ will satisfy \eqref{eq1},
\begin{align} 
a_0 + a_1\vec{A} + a_2\vec{A^{2}} + \cdots+a_n\vec{A^{n}} = 0 \label{eq2}
\end{align}
\section{Proof}
If adj($\vec{A}$) is the adjoint matrix of the matrix $\vec{A}$ of order $n$ which is the transpose of the cofactors of the matrix $\vec{A}$ then, 
\begin{align} 
\vec{A}(adj(\vec{A})) &= \det(\vec{A})\label{eqreplace}
\end{align}
Replacing $\vec{A}$ with ($\vec{A}-\lambda\vec{I}$) in \eqref{eqreplace} we obtain,
\begin{align}
(\vec{A}-\lambda\vec{I})adj(\vec{A}-\lambda\vec{I}) &= \det(\vec{A}-\lambda\vec{I})\vec{I}\label{eq3}
\end{align}
As $\vec{A}$ has a polynomial of degree $n$ for variable $\lambda$, then
$adj(\vec{A}-\lambda\vec{I})$ has a polynomial of degree $n-1$ for variable $\lambda$. Expanding $adj(\vec{A}-\lambda\vec{I})$ with coefficients $b_0,b_1,\cdots,b_n-1$ we get,
\begin{align} 
& adj(\vec{A}-\lambda\vec{I}) = b_0 + b_1\lambda + b_2\lambda^{2} +  \cdots+b_{n-1}\lambda^{n-1}
\end{align}
Hence, from \eqref{eq3}, putting the value of $adj(\vec{A}-\lambda\vec{I})$ we get,
\begin{align}
& (\vec{A}-\lambda\vec{I}) adj(\vec{A}-\lambda\vec{I}) \\&= (\vec{A}-\lambda\vec{I})(b_0 + b_1\lambda + b_2\lambda^{2} +  \cdots+b_{n-1}\lambda^{n-1})\\
&= \vec{A}b_0 + \vec{A}b_1\lambda +\cdots+ \vec{A}b_{n-1}\lambda^{n-1}- b_0\lambda\\ 
&    - b_1\lambda^{2}-\cdots - b_{n-1}\lambda^{n}\\
&=\vec{A}b_0 + \lambda(\vec{A}b_1 - b_0) + \lambda^{2}(\vec{A}b_2 - b_1) + \cdots - b_{n-1}\lambda^{n}  \label{eq4}
\end{align}
Putting values from \eqref{eq1} and \eqref{eq4}  in \eqref{eq3},
\begin{align} 
&\vec{A}b_0 + \lambda(\vec{A}b_1 - b_0) + \lambda^{2}(\vec{A}b_2 - b_1) + \cdots - b_{n-1}\lambda^{n} \\=
&a_0 + a_1\lambda + a_2\lambda^{2} +  \cdots+a_n\lambda^{n}\label{eqMain}
\end{align}
Comparing coefficients of equal powers of $\lambda$ in both sides of \eqref{eqMain},
\begin{align}
\vec{A}b_0 &= a_0 \label{eqS1}\\
\vec{A}b_1 - b_0 &= a_1 \label{eqS2}
\end{align}
\begin{align*}
&\vdots   
\end{align*}
\begin{align}
\vec{A}b_{n-1} - b_{n-2} &= a_{n-1}\label{eqSn_1}\\
-b_{n-1} &= a_n \label{eqSn}
\end{align}
Now, multiplying both sides of \eqref{eqS1} by $\vec{I}$, both sides of \eqref{eqS2} by $\vec{A}$ and so on upto both sides of \eqref{eqSn_1} by $\vec{A^{n-1}}$ and  both sides of \eqref{eqSn} by $\vec{A^{n}}$ we obtain the following,
\begin{align}
\vec{A}b_0 &= a_0\vec{I}\\
\vec{A^{2}}b_1-\vec{A}b_0 &= \vec{A}a_1
\end{align}
\begin{align*}
&\vdots   
\end{align*}
\begin{align}
\vec{A^{n}}b_n-1-\vec{A^{n-1}}b_n-2 &= a_n-1\vec{A^{n-1}}\\
-\vec{A^{n}}b_n-1 &= a_n\vec{A^{n}}
\end{align}
Adding the equations,
\begin{align}
&\vec{A}b_0+\cdots-\vec{A^{n}}b_n-1=a_0 + a_1\vec{A}+\cdots+a_n\vec{A^{n}}\\
\implies &a_0 + a_1\vec{A} + a_2\vec{A^{2}} +  \cdots+a_n\vec{A^{n}} = 0 \label{eqProved}
\end{align}
\eqref{eqProved} together with \eqref{eq1} proves the statement of Cayley-Hamilton theorem.
\\\end{document}
