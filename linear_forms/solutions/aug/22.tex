From \eqref{aug/22/eq:1} and \eqref{aug/22/eq:2}, we get the directional vector of $ L_1 $ and $ L_2 $ as
\begin{align}
    \vec{a} = \myvec{1 \\ -1 \\ -2} , 
    \vec{b} =\myvec{3 \\ -5 \\ -4} \label{aug/22/eq:4}
\end{align}
Angle between the pair of lines is calculated by using cosine formula
\begin{align}
\cos \theta =\frac{\vec{a}^T\vec{b}}{\norm{\vec{a}}\norm{\vec{b}}}
\end{align}
$\because$
\begin{align}
\vec{a}^T \vec{b} =\myvec{1 &-1 & -2} \myvec{3\\-5\\-4} =3+5+8 &=16 \label{aug/22/eq:atb} \\
\norm{\vec{a}} =\sqrt{1^2+(-1)^2+(-2)^2} &=\sqrt{6} \label{aug/22/a}\\
\norm{\vec{b}} =\sqrt{(3)^2+(-5)^2+(-4)^2} &=\sqrt{50} \label{aug/22/b},
\end{align}
%From  \eqref{aug/22/eq:atb} \eqref{aug/22/a} and \eqref{aug/22/b} we get
\begin{align}
\cos \theta &=\frac{16}{\sqrt{6}\sqrt{50}} \\
\implies \theta &= \cos^{-1}{\frac{16}{\sqrt{6}\sqrt{50}}} = 22.5178253587 \degree
\end{align}