
    \item Substituting $\vec{x}=\myvec{x\\0}$,
\begin{align}
    \myvec{1&-1}\myvec{x\\0}&=2\\
    \implies  x&=2
\end{align}
Also, substituting $\vec{x}=\myvec{0\\y}$,
\begin{align}
\myvec{1&-1}\myvec{0\\y}&=2\\
\implies y&=-2
\end{align}
Thus, 
\begin{align}
   \vec{P}=\myvec{2\\0},\vec{Q}=\myvec{0\\-2} 
\end{align}
\item Since the constant $c = 0$,
the line  passes through the origin
Substituting  $\vec{x}=\myvec{1\\y}$ in the equation,
\begin{align}
    \vec{A}=\myvec{0\\0},\vec{B}=\myvec{1\\1}
\end{align}
See Fig. \ref{sep/2/3/beGraphical solution}.
\begin{figure}[ht]
    \centering
    \includegraphics[width=\columnwidth]{solutions/sep/2/3/be/Python Img.png}
    \caption{Graphs of Equations (a) and (b)}
    \label{sep/2/3/beGraphical solution}
\end{figure}
